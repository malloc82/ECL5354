\textbf{5.} The autocorrelation of a zero mean voltage noise source, $V(t)$, is
  $$ R_V (\tau ) = 10 \; \exp(-|\tau|)\mbox{ volt}^2.$$
  Time $t$ is measured in seconds and frequency $f$ in Hertz. We are going to analyze this voltage source in three different scenarios.
  \begin{enumerate}
  \item $V(t)$ is applied across a 5 ohm resister. What is the average power dissipated by the resister?


  \item  The voltage $V(t)$ is applied across a series connection of a 2 ohm resistance and a one farad capacitor. Call the voltage across the resister $U(t)$.
    \begin{enumerate}
    \item What is the power spectral density of $U(t)$?

    \item What is the autocorrelation of the noise across the resister?

    \item What is the average power dissipated by the resister?


    \end{enumerate}
  \item The voltage $V(t)$ is passed through a low pass voltage filter that discards all frequency components higher that 4 Hertz and keeps the rest. The voltage at the filter output is applied across a four ohm resister. What power is dissipated by the resister?
  \end{enumerate}

